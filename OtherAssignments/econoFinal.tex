% Options for packages loaded elsewhere
\PassOptionsToPackage{unicode}{hyperref}
\PassOptionsToPackage{hyphens}{url}
%
\documentclass[
  11pt,
]{article}
\usepackage{amsmath,amssymb}
\usepackage[]{mathpazo}
\usepackage{iftex}
\ifPDFTeX
  \usepackage[T1]{fontenc}
  \usepackage[utf8]{inputenc}
  \usepackage{textcomp} % provide euro and other symbols
\else % if luatex or xetex
  \usepackage{unicode-math}
  \defaultfontfeatures{Scale=MatchLowercase}
  \defaultfontfeatures[\rmfamily]{Ligatures=TeX,Scale=1}
\fi
% Use upquote if available, for straight quotes in verbatim environments
\IfFileExists{upquote.sty}{\usepackage{upquote}}{}
\IfFileExists{microtype.sty}{% use microtype if available
  \usepackage[]{microtype}
  \UseMicrotypeSet[protrusion]{basicmath} % disable protrusion for tt fonts
}{}
\makeatletter
\@ifundefined{KOMAClassName}{% if non-KOMA class
  \IfFileExists{parskip.sty}{%
    \usepackage{parskip}
  }{% else
    \setlength{\parindent}{0pt}
    \setlength{\parskip}{6pt plus 2pt minus 1pt}}
}{% if KOMA class
  \KOMAoptions{parskip=half}}
\makeatother
\usepackage{xcolor}
\usepackage[margin = 1in]{geometry}
\usepackage{graphicx}
\makeatletter
\def\maxwidth{\ifdim\Gin@nat@width>\linewidth\linewidth\else\Gin@nat@width\fi}
\def\maxheight{\ifdim\Gin@nat@height>\textheight\textheight\else\Gin@nat@height\fi}
\makeatother
% Scale images if necessary, so that they will not overflow the page
% margins by default, and it is still possible to overwrite the defaults
% using explicit options in \includegraphics[width, height, ...]{}
\setkeys{Gin}{width=\maxwidth,height=\maxheight,keepaspectratio}
% Set default figure placement to htbp
\makeatletter
\def\fps@figure{htbp}
\makeatother
\setlength{\emergencystretch}{3em} % prevent overfull lines
\providecommand{\tightlist}{%
  \setlength{\itemsep}{0pt}\setlength{\parskip}{0pt}}
\setcounter{secnumdepth}{-\maxdimen} % remove section numbering
\newlength{\cslhangindent}
\setlength{\cslhangindent}{1.5em}
\newlength{\csllabelwidth}
\setlength{\csllabelwidth}{3em}
\newlength{\cslentryspacingunit} % times entry-spacing
\setlength{\cslentryspacingunit}{\parskip}
\newenvironment{CSLReferences}[2] % #1 hanging-ident, #2 entry spacing
 {% don't indent paragraphs
  \setlength{\parindent}{0pt}
  % turn on hanging indent if param 1 is 1
  \ifodd #1
  \let\oldpar\par
  \def\par{\hangindent=\cslhangindent\oldpar}
  \fi
  % set entry spacing
  \setlength{\parskip}{#2\cslentryspacingunit}
 }%
 {}
\usepackage{calc}
\newcommand{\CSLBlock}[1]{#1\hfill\break}
\newcommand{\CSLLeftMargin}[1]{\parbox[t]{\csllabelwidth}{#1}}
\newcommand{\CSLRightInline}[1]{\parbox[t]{\linewidth - \csllabelwidth}{#1}\break}
\newcommand{\CSLIndent}[1]{\hspace{\cslhangindent}#1}
\usepackage{setspace}\doublespacing
\usepackage[left]{lineno}
\linenumbers
\usepackage{dcolumn}
\usepackage{caption}
\usepackage{float}
\usepackage{afterpage}
\usepackage{siunitx}
\usepackage{amsmath}
\ifLuaTeX
  \usepackage{selnolig}  % disable illegal ligatures
\fi
\IfFileExists{bookmark.sty}{\usepackage{bookmark}}{\usepackage{hyperref}}
\IfFileExists{xurl.sty}{\usepackage{xurl}}{} % add URL line breaks if available
\urlstyle{same} % disable monospaced font for URLs
\hypersetup{
  pdftitle={Operating Characteristics and Test Information of the Ages and Stages Questionnaire},
  pdfauthor={Adon Rosen},
  pdfkeywords={Item Response Theory, Test Information, Classification,
Developmental Monitoring},
  hidelinks,
  pdfcreator={LaTeX via pandoc}}

\title{\textbf{Operating Characteristics and Test Information of the
Ages and Stages Questionnaire}}
\author{Adon Rosen}
\date{December 14, 2022}

\begin{document}
\maketitle

\hypertarget{introduction}{%
\section{Introduction}\label{introduction}}

Developmental and behavioral difficulties (DBDs) cost society an
estimated \$250 billion per year (Bradley, 2003). With DBD prevalence on
the rise, the economic gradient is steepening (Boyle et al., 2011). When
unnoticed and untreated, the price of DBDs and the number of ensuing
negative impacts increases. Early interventions have proven effective at
remediating and preventing many DBDs, but resources for early
identification are limited (Rosenberg, Zhang, \& Robinson, 2008).
Screening tools such as the Ages and Stages Questionnaire (ASQ) are
frequently employed to identify individuals at-risk of DBDs (Bricker \&
Squires, 1989). Screening tools, such as the ASQ, seek to balance ease
of implementation, as well as sensitivity to the presence of DBDs in
order successfully identify children at-risk of DBDs. Psychometric
validation is commonly used on screener tools to validate the
sensitivity of these tools; however, these validation techniques
typically employ factor analytic strategies which ignore the
implementation of the tools, or use reliability assessments which are
insensitive to the tails of the distribution (children at-risk of having
DBDs). This study seeks to explore the psychometric properties of the
ASQ by proposing a technique which is sensitive to the scale's ability
to identify individual's who are at-risk of possessing DBDs.

The ASQ is comprised of five main subscales evaluating development in
Communication, Gross Motor, Fine Motor, Problem Solving, and
Personal-Social abilities (Squires, Twombly, Bricker, \& LaWanda, 2009).
Within each subscale, there are 20 versions of the ASQ, the version a
child receives is determined by the child's age in months. Every version
has six partial credit questions. As intentioned, these screening items
target behaviors and skills at the middle to low-end of each ability
scale. Most current psychometrics are based on a sample of 15,138 unique
children who completed 18,572 questionnaires (3,434 individuals
completed 1+). Two-week test-retest correlations on the order of .75 to
.82 (n=145), inter-observer correlations ranging from 0.43 to 0.69
(n=107), and internal consistency correlations between 0.60 and 0.84
support reliability. There is also relatively good support for construct
validity of the measure's categorical ratings of risk with sensitivity
and specificity estimates exceeding 0.83 when discriminating at-risk
samples from the general population. Additional explorations of the ASQ
have also used measurement models such as factor analysis to confirm the
performance of the tool.

When exploring the factor structure of the ASQ commonly exploratory
factor analytic techniques have been applied. The goal of applying an
exploratory model tests the belief that every factor scale will map onto
the subscale of interest. Recent explorations of the factor structure of
the ASQ has supported the stability of the items within the subscale and
over time (Olvera Astivia, Forer, Dueker, Cowling, \& Guhn, 2017). While
the application of an exploratory model confirms the items load onto the
latent trait, a more applicable model for how the ASQ is being
implemented would be a confirmatory factor model (CFA). Through a CFA
framework, it enforces the items to act onto the latent trait they are
being to used to test. A popular CFA model to test the quality of
dichotomous or graded response items, such as those that the ASQ
employs, is an IRT model (Embretson \& Reise, 2000).

An IRT model can be described as:
\[P_i(\theta)=\frac{1}{1+e^{-a_i(\theta-b_i)}}\]

In this formula, \(P_i\) represents the probability an item is endorsed,
\(\theta\) is the ability estimate of an individual, \(a_i\) represents
the discrimination estimate for item \(i\), and \(b_i\) represents the
difficulty estimate for item \(i\). The above characteristics can be
used to map the item characteristic curve (ICC) a graphical
representation of the item's characteristics. An IRT model which
includes partial credit is typically called a graded response model
(GRM). The GRM extends the formulation of the binary IRT model by
including probabilities for each potential endorsement:
\[P(x_{i}= k|\theta)=\frac{1}{1+e^{-a_i(\theta-b_{ik})}} - \frac{1}{1+e^{-a_i(\theta-b_{i(k+1)})} }\]
This formula now includes a difficulty parameter and endorsement
probability for specific response values, \(k\) for every item \(i\). By
estimating the characteristics using this GRM framework both the
implementation, and the underlying theory of the ASQ are more closely
adhered. Through the model estimation, by estimating only a
discrimination parameter across the entire scale this mimics the
sum-score approach employed in the ASQ (McNeish \& Wolf, 2020). By using
the CFA framework, it also adheres to the fact that every item only maps
onto a single subscale. Finally, the greatest benefit of applying an IRT
estimation framework is the ability to estimate quality of an item and a
scale for an indivdual.

Within the IRT framework, ``quality'' is determined by the amount of
information an item possess for an individual. The amount of infomration
an item possess is estimated by an indivudla's esitmated ability and an
item's characteristics. These item characteristics are used to calculate
an item's information at specific levels of theta (Louis, 1982; Monroe,
2019). The information for an item adheres to the following formula:
\[I_i(\theta)=a^2_iP_i(\theta)Q_i(\theta)\]

Where \(I_i(\theta)\) is the information produced by a item \(i\),
\(a_i\) is the item's discrimination, \(P_i(\theta)\) is the probability
for endorsement, and \(Q_i(\theta)\) is the probability of non
endorsement. This formulation indicates several characteristics about
the amount of information an item possess: first, as \(a_i\) increases
the amount of information increases, a larger discrimination parameter
is almost always desirable, second, information is the greatest amount
of information exists at the inflection points of an item's
characteristic curve, this means that information is maximized when an
examinee has a 50\% chance of endorsement. Finally, information varies
per person.

The test information function sums across every item's information
function and takes the following form:
\[I_t(\theta)=\sum_{i=1}^NI_i(\theta)\]

Where \(I_t(\theta)\) represents the amount of information a test
possess at a specific ability level. Finally, the total relevant
information takes the integral of the test information function based on
the cutoff values taken at the latent ability level. Motivated by
methodology of the ASQ, the scoring cutoff explores any individual two
standard deviations below the mean score, so the total relevant
information for a high-risk classification from an ASQ administration
takes the following form:
\[I_r=\int_{-\infty}^{-2}\sum_{i=1}^NI_i(\theta)d(\theta)\]

Test information is frequently used in computerized adaptive testing
where the goal is to maximize the amount of information for an examinee
by modifying the item bank for every individual (Moore et al., 2019). By
extending this methodology to a classification problem, it allows
insights into the tests ability to identify individual's within a
predetermined \(\theta\) range. This alternative estimation of test
reliability mitigates the concerns of alternative reliability estimates
which explore scale reliability over the entire \(\theta\) range. By
applying these methods, and estimating the total relevant information to
an developmental screener it allows researchers an assessment to the
quality of the tool, and the likelihood of misclassification of a DBD.

\hypertarget{methods}{%
\section{Methods}\label{methods}}

The goal of this study is to explore if total relevant information from
the ASQ-3 could be used to identify high-risk classification
inconsistencies among real children. In order to perform this, several
tasks had to be completed. First, the reliability and intra-class
correlation of ASQ administrations were explored as a function of time.
Second, a IRT estimates were estimated following a GRM formulation for
every version within each subscale of the ASQ-3. Third, the total
relevant information was estimated using these GRM parameter estimates.
Fourth and finally, relationships between total relevant information and
classification inconsistencies are explored.

\hypertarget{participants}{%
\subsubsection{Participants}\label{participants}}

Parents receiving home-based parenting services were identified via the
Oklahoma State Department of Health's Efforts-to-Outcomes administrative
database and contacted using mailed letters, phone calls, in-person, and
electronic communication procedures. Parents who wished to participate
were provided with additional study information and research assistants
scheduled a time to complete the interview. Surveys primarily took place
in the home with the index child present. Upon participant request, some
surveys were conducted in a private location other than the home, such
as a library, or virtually due to COVID-19 restrictions. Research
assistants obtained informed consent, then the survey was administered
through REDCap, a secure web-based application for data collection
(Harris et al., 2009). The survey consisted of a battery of measures
aimed at parent and child health and well-being, parent-child
interactions, and child development. Specific measures used for the
present analyses are described in detail below. Participants' responses
were confidential; however, research assistants were available to answer
questions or help with survey administration if requested. Surveys
lasted approximately 1-3 hours, and participants were compensated for
their time.

\hypertarget{asq-reliability}{%
\subsubsection{ASQ reliability}\label{asq-reliability}}

The first set of analyses explores how consistent the reliability of the
ASQ-3 is as the time between administrations increases. By exploring the
variability of reliability as a function of time it can inform
downstream analyses if the differences in identification are driven by
diminishing reliability. In order to assess this, an initial caliper of
30 days was applied, any serial ASQ administrations that were performed
within the caliper had a Krippendorff's \(\alpha\) and an ICC
calculated. The ICC was estimated assuming multiple random raters as the
ASQ may have been administered to any of the child's care providers. The
caliper was extended by increments of 5, ranging from 30 to a total of
200 days separation in the administration. This was performed separately
for the \emph{z}-scored ASQ scores as well as the risk-categories
assigned in an ASQ administration. For the \emph{z}-scored values the
Interval approached was used to estimate the Krippendorff's \(\alpha\)
whereas the ordinal approach was used for the risk-category.

\hypertarget{total-relevant-information-for-the-asq-3}{%
\subsubsection{Total Relevant Information for the
ASQ-3}\label{total-relevant-information-for-the-asq-3}}

In order to obtain a domain and version specific total relevant
information two tasks had to be performed. First a GRM parameters have
to be estimated for every version and subscale, second, these GRM
characteristics must be used to identify the test information functions,
third and finally, these test information functions are used to identify
the total relevant information. The estimation of an IRT model was
motivated by the ASQ scoring practices. That is, because sum-scores are
utilized in an ASQ, a fixed discrimination parameter was estimated
within each version within each domain of the ASQ (McNeish \& Wolf,
2020). Accordingly, when an IRT model was estimated a single
discrimination parameter was estimated as well as two difficulty
parameters for every item for a partial and complete endorsement of an
item. This requires 12 difficulty parameters and a single discrimination
parameter to be estimated. These models were estimated using the
\texttt{mirt} (Chalmers, 2012) package in R version 4.2.1 (R Core Team,
2020). When no endorsements of an item were present, or complete full
endorsement of an item was present, the item's difficulty was assigned
to -4, or 4 respectively.

Next, the test information function is estimated using the parameter
estimates identified within every domain and version. The test
information function was calculated using \texttt{testinfo} function
from the \texttt{mirt} package. The \texttt{testinfo} returns a series
of point estimates of information at predetermined levels of theta, in
order to calculate the integral of this function the \texttt{integrate}
function was used from \texttt{R}. The integrated between \(-\infty\)
and \(-2\) was calculated for every domain and version for these test
information functions.

\hypertarget{modeling-high-risk-inconsistencies-between-serial-asq-administrations}{%
\subsubsection{Modeling High-risk Inconsistencies Between Serial ASQ
Administrations}\label{modeling-high-risk-inconsistencies-between-serial-asq-administrations}}

In order to model the high-risk inconsistencies between serial
administrations of an ASQ as a function of relevant information, several
tasks had to be performed. First, participants who had received serial
administrations of an ASQ within 100 days were identified. Next, the
outcome, agreement in high-risk status, was created for every
participant. An agreement was identified if both ASQ administrations
identified the participant as either high-risk or not high-risk. That
is, if an individual had a first administration ASQ risk category of
normal development and the second administration they received a score
within the monitoring range, this was classified as an agreement between
the two ASQ administrations. However, if the first ASQ administration
scored the individual as having normal development and the second scored
them as having a high-risk of developmental delay, this was classified
as a disagreement. To test for agreement, a generalized linear mixed
effects model was estimated with a logistic link function. The outcome
agreement was modeled as a function of the total relevant information
from the first and second ASQ administrations and the interaction of
these two values. The model also controlled for the version of the first
administration and the time between the administrations in days. A
random intercept was included for every participant, as every
participant had at least 5 agreement statuses, one for every ASQ domain.
The model was trained using the \texttt{lme4} package (Bates, Mächler,
Bolker, \& Walker, 2015). The formula takes the following structural
format:

\[
Y_{st}=\beta_{0s}+\beta_{1} \times X_{TRI1st} + \beta_2 \times X_{TRI2st} + \beta_3 \times X_{TRI1st*TRI2st}+ \beta_{c} \times X_{c} + e_{st}
\]

Where \(Y_{st}\) represents high-risk agreement, \(\beta_{0s}\) is an
individual specific intercept term, \(\beta_{1}, \beta_{2}\) are the
fixed effects for the total relevant information, with
\(X_{TRI1st},X_{TRI2st}\) are the the total relevant information values
for individual \(s\) at time \(t\). The \(\beta_c\) and \(X_c\)
represent fixed effects for covariates such as time between
administration and child age at initial administration. Finally,
\(e_{st}\) represents an individual specific error term.

\hypertarget{counter-factual-examination-of-at-risk-inconsistency}{%
\subsubsection{Counter Factual Examination of at-risk
Inconsistency}\label{counter-factual-examination-of-at-risk-inconsistency}}

Finally, based on the results from the inconsistency analysis an optimal
administration timeline will be explored. By using a counter factual
exploration it will allow for the time between serial administrations
dependent on child's age and an ASQ score to be estimated. The following
structural model will be estimated

\[
y_{st}^*=\lambda_v (\gamma'x+\zeta) + e_{st}
\] where \(y_{st}^*\) reflects the predicted probability of agreement
between two serial administrations, \(\lambda_v\) reflects the
discrimination parameters for a specific ASQ version, \(\gamma\) is a
vector of regression coefficients, \(x_{st}\) is a vector of manifest
variables, \(\zeta_{st}\) is a disturbance term and \(e_{st}\) is a
residual variable. This model formulation adheres to a multiple
indicator multiple causes model (Muthén, 1984). The goal of these
analyses is to identity the optimal time between administrations which
minimizes the risk of a false positive. In order to perform this,
parameter estimates will be obtained using the entire population and
individual traits will be modified for an individual's manifest
variables including their ASQ score, and their time between
administration. By varying an individual's ASQ score it allows for an
examination into the relationship between the standard error of
measurement and misclassification. Specifically, individual's who may
have a version with lower information, and therefore a greater standard
error of measurement, are more likely to exhibit missclassification when
the score is closer to an at-risk threshold. By varying this in
conjunction with the time between administration it will guide the best
time to perform a follow up examination to confirm a at-risk status.

\hypertarget{anticipated-results}{%
\section{Anticipated Results}\label{anticipated-results}}

\hypertarget{asq-reliability-1}{%
\subsubsection{ASQ Reliability}\label{asq-reliability-1}}

As time between administrations of an ASQ increase, it is estimated that
the reliability of the scale should decrease. Because the ASQ seeks to
measure an individual's developmental status, and development is a
moving target littered with individualized trajectories overtime. An
additional concern for the reliability of the scale is the amount of
information changes over ASQ versions. SO the quality of the exam will
vary overtime. These things altogether contribute to a diminishing
reliability of the ASQ as a function of time.

\hypertarget{modeling-high-risk-inconsistencies-between-serial-asq-administrations-1}{%
\subsubsection{Modeling High-risk Inconsistencies Between Serial ASQ
Administrations}\label{modeling-high-risk-inconsistencies-between-serial-asq-administrations-1}}

It is anticipated that as the difference between the total relevant
information between serial examinations increases, the probability of an
inconsistency will increase. As the total relevant information is an
assessment of scale quality, when the difference between the quality of
the scale increases it becomes more likely that the scale will identify
differences.

\hypertarget{counter-factual-examination-of-at-risk-inconsistency-1}{%
\subsubsection{Counter Factual Examination of at-risk
Inconsistency}\label{counter-factual-examination-of-at-risk-inconsistency-1}}

The counter factual examination seeks to identify the likelihood an
individual is at-risk based on their score. In order to perform this the
time between administration will be varied as well as an individual's
ASQ score. The motivation of these analyses is to increase the
confidence in an individual's at-risk assessment while also lowering the
burden in downstream referral to developmental promotion providers. It
is theorized that when an individual receives a score further from the
at-risk threshold (i.e.~\(\theta < -2\)) than their time between
administration to confirm the are at-risk of DBDs. However, when an
individual is closer to the threshold then a serial assessment may
reveal they graduate to a lower risk assessment. The anticipated results
will reveal a inverse quadratic relationship, as an individual is
further from the threshold serial assessments to confirm the original
status will need to be closer and when an individual is closer to the
threshold serial assessments should be further separated to allow time
for development between assessments.

\hypertarget{bibliography}{%
\section*{Bibliography}\label{bibliography}}
\addcontentsline{toc}{section}{Bibliography}

\hypertarget{refs}{}
\begin{CSLReferences}{1}{0}
\leavevmode\vadjust pre{\hypertarget{ref-bates2015}{}}%
Bates, D., Mächler, M., Bolker, B., \& Walker, S. (2015). Fitting Linear
Mixed-Effects Models Using lme4. \emph{Journal of Statistical Software},
\emph{67}, 1--48.
doi:\href{https://doi.org/10.18637/jss.v067.i01}{10.18637/jss.v067.i01}

\leavevmode\vadjust pre{\hypertarget{ref-boyle2011}{}}%
Boyle, C. A., Boulet, S., Schieve, L. A., Cohen, R. A., Blumberg, S. J.,
Yeargin-Allsopp, M., \ldots{} Kogan, M. D. (2011). Trends in the
prevalence of developmental disabilities in US children, 1997-2008.
\emph{Pediatrics}, \emph{127}(6), 1034--1042.
doi:\href{https://doi.org/10.1542/peds.2010-2989}{10.1542/peds.2010-2989}

\leavevmode\vadjust pre{\hypertarget{ref-bradley2003}{}}%
Bradley, S. J. (2003). Handbook of early childhood intervention. Second
edition. \emph{The Canadian Child and Adolescent Psychiatry Review},
\emph{12}(4), 122. Retrieved from
\url{https://www.ncbi.nlm.nih.gov/pmc/articles/PMC2533836/}

\leavevmode\vadjust pre{\hypertarget{ref-bricker1989}{}}%
Bricker, D., \& Squires, J. (1989). The Effectiveness of Parental
Screening of At-Risk Infants: The Infant Monitoring Questionnaires.
\emph{Topics in Early Childhood Special Education}, \emph{9}(3), 67--85.
doi:\href{https://doi.org/10.1177/027112148900900306}{10.1177/027112148900900306}

\leavevmode\vadjust pre{\hypertarget{ref-chalmers2012a}{}}%
Chalmers, R. P. (2012). mirt: A Multidimensional Item Response Theory
Package for the R Environment. \emph{Journal of Statistical Software},
\emph{48}, 1--29.
doi:\href{https://doi.org/10.18637/jss.v048.i06}{10.18637/jss.v048.i06}

\leavevmode\vadjust pre{\hypertarget{ref-embretson2000a}{}}%
Embretson, S. E., \& Reise, S. P. (2000). \emph{Item response theory}.
New York: Psychology Press.
doi:\href{https://doi.org/10.4324/9781410605269}{10.4324/9781410605269}

\leavevmode\vadjust pre{\hypertarget{ref-harris2009}{}}%
Harris, P. A., Taylor, R., Thielke, R., Payne, J., Gonzalez, N., \&
Conde, J. G. (2009). Research electronic data capture (REDCap)--a
metadata-driven methodology and workflow process for providing
translational research informatics support. \emph{Journal of Biomedical
Informatics}, \emph{42}(2), 377--381.
doi:\href{https://doi.org/10.1016/j.jbi.2008.08.010}{10.1016/j.jbi.2008.08.010}

\leavevmode\vadjust pre{\hypertarget{ref-louis1982}{}}%
Louis, T. A. (1982). Finding the observed information matrix when using
the EM algorithm. \emph{Journal of the Royal Statistical Society. Series
B (Methodological)}, \emph{44}(2), 226--233. Retrieved from
\url{https://www.jstor.org/stable/2345828}

\leavevmode\vadjust pre{\hypertarget{ref-mcneish2020}{}}%
McNeish, D., \& Wolf, M. G. (2020). Thinking twice about sum scores.
\emph{Behavior Research Methods}, \emph{52}(6), 2287--2305.
doi:\href{https://doi.org/10.3758/s13428-020-01398-0}{10.3758/s13428-020-01398-0}

\leavevmode\vadjust pre{\hypertarget{ref-monroe2019}{}}%
Monroe, S. (2019). Estimation of Expected Fisher Information for IRT
Models. \emph{Journal of Educational and Behavioral Statistics},
\emph{44}(4), 431--447.
doi:\href{https://doi.org/10.3102/1076998619838240}{10.3102/1076998619838240}

\leavevmode\vadjust pre{\hypertarget{ref-moore2019}{}}%
Moore, T. M., Calkins, M. E., Satterthwaite, T. D., Roalf, D. R., Rosen,
A. F., Gur, R. C., \& Gur, R. E. (2019). Development of a computerized
adaptive screening tool for overall psychopathology ({``}p{''}).
\emph{Journal of Psychiatric Research}, \emph{116}, 2633.

\leavevmode\vadjust pre{\hypertarget{ref-muthuxe9n1984}{}}%
Muthén, B. (1984). A general structural equation model with dichotomous,
ordered categorical, and continuous latent variable indicators.
\emph{Psychometrika}, \emph{49}(1), 115--132.
doi:\href{https://doi.org/10.1007/BF02294210}{10.1007/BF02294210}

\leavevmode\vadjust pre{\hypertarget{ref-olveraastivia2017}{}}%
Olvera Astivia, O. L., Forer, B., Dueker, G. L., Cowling, C., \& Guhn,
M. (2017). The Ages and Stages Questionnaire: Latent factor structure
and growth of latent mean scores over time. \emph{Early Human
Development}, \emph{115}, 99--109.
doi:\href{https://doi.org/10.1016/j.earlhumdev.2017.10.002}{10.1016/j.earlhumdev.2017.10.002}

\leavevmode\vadjust pre{\hypertarget{ref-rcoreteam2020a}{}}%
R Core Team. (2020). \emph{R: A language and environment for statistical
computing}. Vienna, Austria. Retrieved from
\url{https://www.R-project.org/}

\leavevmode\vadjust pre{\hypertarget{ref-rosenberg2008}{}}%
Rosenberg, S. A., Zhang, D., \& Robinson, C. C. (2008). Prevalence of
developmental delays and participation in early intervention services
for young children. \emph{Pediatrics}, \emph{121}(6), e1503--1509.
doi:\href{https://doi.org/10.1542/peds.2007-1680}{10.1542/peds.2007-1680}

\leavevmode\vadjust pre{\hypertarget{ref-squires2009}{}}%
Squires, J., Twombly, E., Bricker, D., \& LaWanda, P. (2009).
\emph{ASQ®-3 user's guide}. Brookes. Retrieved from
\url{https://products.brookespublishing.com/ASQ-3-Users-Guide-P571.aspx}

\end{CSLReferences}

\end{document}
