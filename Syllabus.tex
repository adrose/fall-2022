\documentclass[11pt,english]{article}
\usepackage{mathpazo}
\renewcommand{\familydefault}{\rmdefault}
\usepackage[T1]{fontenc}
\usepackage[latin9]{inputenc}
\usepackage{geometry}
\geometry{verbose,tmargin=1in,bmargin=1in,lmargin=1in,rmargin=1in}
\usepackage{color}
\usepackage[dvipsnames]{xcolor}
\definecolor{oucrimson}{RGB}{132.,22.,23.}
\usepackage[]{threeparttable}
\usepackage{babel}
\usepackage{booktabs}
\usepackage{array}
\usepackage[authoryear]{natbib}
\usepackage[unicode=true,pdfusetitle,bookmarks=true,bookmarksnumbered=false,bookmarksopen=false,breaklinks=true,pdfborder={0 0 0},pdfborderstyle={},backref=false,colorlinks=true]{hyperref}
\hypersetup{citecolor=oucrimson,filecolor=oucrimson,linkcolor=oucrimson,urlcolor=oucrimson}
\usepackage{breakurl}

\usepackage{pdflscape}

\begin{document}

\title{Introduction to Statistics}

\author{University of Oklahoma\\}
\date{}

\maketitle
\section*{\rule[0.5ex]{1\columnwidth}{0.5pt}}

\begin{flushleft}
\begin{tabular}{rl}
\textbf{Instructor:}          & Adon Rosen                                       \\
\textbf{Email:}               & \href{mailto:adon.rosen@ou.edu}{adon.rosen@ou.edu}         \\
\textbf{Office Hours:}        & TBA  \\
                              &                                                    \\
                              &                                                    \\
\textbf{Teaching Assistant:}  & TBA                                                \\
\textbf{Office:}              & TBA                                                \\
\textbf{Office Hours:}        & TBA                                                \\
\end{tabular}
\par\end{flushleft}

\section*{\rule[0.5ex]{1\columnwidth}{0.5pt}}

\section*{Course Description}

Introduction to statistics 

\section*{Student Learning Outcomes}
This course seeks to teach students several distinct skills, these include: 
\begin{enumerate}
\item Understanding when and how, and which descriptivie statistics to use
\item Understand the basics of inference testing
\item Interpreting graphical methods for describing inference tests
\item Drawing practical importnace from statisical conclusions
\item Ethidcal considerations when describing data
\item Understanding good coding practices of languages such as R or Python
\end{enumerate}
For further detail on course content, see the course schedule at the end of this document. This is a 3-credit-hour course, which means we will have about 3 hours per week of class. You should expect to spend, on average, another 4-6 hours per week outside of class on reading, preparation, homework, and review.\footnote{From OU's ``How to Graduate a Sooner'' webpage: ``On average, you should expect to spend 2-3 hours outside of class studying for each credit hour you are taking.'' (\url{http://www.ou.edu/graduatesooner/resources/graduate_a_sooner.html})}


\subsection*{Textbook and other materials}

Materials for the course will be assigned from the following sources:
\begin{enumerate}
    \item The main textbook for the first part of this course will be \emph{Statistical Thinking for the 21st Century}, by Russell A. Poldrack. The book is available at \url{https://github.com/statsthinking21/statsthinking21-core} for free. 
\end{enumerate}

\subsection*{Course website}

Class announcements and homework will be posted on the course website on Canvas: \url{https://canvas.ou.edu}. It is your responsibility to check the site regularly\textemdash at least every day class is held. All important announcements will be posted there.

Most course materials will also be publicly posted on GitHub at \url{https://github.com/IntroStats/}.

\section*{\rule[0.5ex]{1\columnwidth}{0.5pt}}

\section*{Grading policies}

Your grade will be determined by the following criteria:

\begin{itemize}
\item Problem sets (30\%)
    \begin{itemize}
        \item A problem set will be assigned for each chapter covered in this class; lowest score will be dropped. You may work on these in groups but you must turn in an individual copy.
    \end{itemize}
\item Group Work (30\%)
    \begin{itemize}
        \item Groups will design and perofrm a study using data simulated in a coding language of the groups choice. Group work will be graded in a peer-review format, that is groups will grade one another thorughout the semester to provide feedback for study design, analysis, and conclusions. I will spot check the peer reviews throughout to ensure accurate and constructive feedback is being provided. Peer-reviews and also the final result of the project will be combined into the total groupwork project.
    \end{itemize}
\item Midterm exam (20\%)
\item Final exam (20\%)
    \begin{itemize}
        \item There will be one midterm and one final exam. After students recieve their grades, they can earn back half of the missed points by correcting their mistakes and presenting a clear understanding of where and why points were missed to the instructor. 
    \end{itemize}

\end{itemize}


\subsection*{\rule[0.5ex]{1\columnwidth}{0.5pt}}


\subsection*{Grading scale}

All exams will be out of 200 points. At the end of the course, I will compute a final percentage grade based on component percentages of each grade category using the weights given above. I will then convert this final percentage grade into letter grades according to the following scale (where $g$ indicates your final percentage grade): 

\begin{center}
\begin{tabular}{cccc}
90\%$\leq$ & $g$ & $\leq$100\% & A\\
80\%$\leq$ & $g$ & $<$90\%     & B\\
70\%$\leq$ & $g$ & $<$80\%     & C\\
60\%$\leq$ & $g$ & $<$70\%     & D\\
 0\%$\leq$ & $g$ & $<$60\%     & F\\
\end{tabular}
\par\end{center}

I reserve the right to scale upwards everyone's final percentage grades by a common factor (e.g. 1.06), but the course will not be graded on a curve, and no one's final percentage grade will be lowered.

\section*{\rule[0.5ex]{1\columnwidth}{0.5pt}}

\section*{Classroom etiquette}

I value your presence in my class, and I want your classmates to feel the same way. You are welcome to eat/drink during class as long as food/drink is permitted in the classroom and you do not disrupt or distract others by doing so. Note that smoking is prohibited on all OU property. Please silence your cell phones, pagers, or other electronic devices during class, and do not use them in the classroom. If you need to respond to a text/social media message, or make a phone call, please leave the classroom before doing so. 
You should bring your laptop to class (if you have one), as we will do in-class exercises almost every class period. Please do not use your laptop for work that is not directly related to what we are doing in class. Doing so has been scientifically proven to reduce your own academic performance, as well as that of your peers.\footnote{See, for example, \href{https://www.nytimes.com/2017/11/22/business/laptops-not-during-lecture-or-meeting.html}{this NYT column}.}

\section*{Contacting me}
I will always be available during my office hours. Due to Covid, I will only be holding office hours via Zoom. Please sign up to meet with me at a link to be determined.

If you ever need to email me or any other professor at OU, please follow the basic rules contained at the following link: \url{http://www.jamestierney.com/teaching/how-to-email-a-professor/}

I will promise to reply to your email within 48 hours of your sending it. 

\section*{Course Policies}

\subsection*{Make-up Policy}

All work should be turned in on the day it is due. Late work will only be accepted for university-excused absences, illnesses, or other unforeseen emergencies.

\subsection*{Absences}

Absences from class will only be excused for university approved reasons, illnesses, or other unforeseen emergencies.

\subsection*{\rule[0.5ex]{1\columnwidth}{0.5pt}}

\section*{University Policies}


\subsection*{Academic Integrity}

I do not tolerate academic misconduct, and neither does the University of Oklahoma: 
\begin{quotation}
``Academic misconduct is cheating. More precisely, it is any action that a student knows (or should know) will lead to the improper evaluation of academic work. If the professor does not detect it, academic misconduct defeats the purpose of academic work because you are pretending to know more or write better than you actually do. ... 

``At OU, acts of plagiarism can receive institutional penalties ranging from a letter of reprimand to required coursework to expulsion. All academic misconduct offenses also receive grade penalties determined by the instructor. Grade penalties are not restricted to the value of the assignment and may be up to an F in the course. Juniors and seniors who plagiarize any significant portion of a paper should expect at least a suspension for a spring or fall semester. Under the right circumstances even freshmen and sophomores may also receive suspensions or even be expelled for plagiarism.'' 

\textemdash \url{http://integrity.ou.edu/files/nine_things_you_should_know.pdf}
\end{quotation}
For further information on what constitutes academic misconduct, as well as how such misconduct is punished, please consult the Student Guide to Academic Dishonesty, found at the following link:\\ \url{https://integrity.ou.edu/students.html}\\

I will not hesitate to fail students who do not fully comply with the University's academic misconduct policy. If you find yourself contemplating cheating, plagiarism, or other forms of academic misconduct, please come see me first. Help is available if you are struggling. I want everyone in the class to try their best and to do their own work. Please be advised that I reserve the right to utilize anti-plagiarism resources such as \emph{TurnItIn} when grading assignments.

\subsection*{Religious Observance}

It is the policy of the University to excuse the absences of students that result from religious observances and to reschedule examinations and additional required classwork that may fall on religious holidays, without penalty.

\subsection*{Reasonable Accommodation Policy}

The Accessibility and Disability Resource Center is committed to supporting students with disabilities to ensure that they are able to enjoy equal access to all components of their education.  This includes your academics, housing, and community events.  If you are experiencing a disability, a mental/medical health condition that has a significant impact on one or more life functions, you can receive accommodations to provide equal access.  Possible disabilities include, but are not limited to, learning disabilities, AD(H)D, mental health, and chronic health.  Additionally, we support students with temporary medical conditions (broken wrist, shoulder surgery, etc.) and pregnancy.  To discuss potential accommodations, please contact the ADRC at 730 College Avenue, (ph.) 405.325.3852, or \href{mailto:adrc@ou.edu}{\texttt{adrc@ou.edu}}. 

\subsection*{Title IX Resources and Reporting Requirement}

Anyone who has been impacted by gender-based violence, including dating violence, domestic violence, stalking, harassment, and sexual assault, deserves access to resources so that they are supported personally and academically. The University of Oklahoma is committed to offering resources to those impacted, including: speaking with someone confidentially about your options, medical attention, counseling, reporting, academic support, and safety plans. If you would like to speak with someone confidentially, please contact \href{https://www.ou.edu/gec/gender-based-violence/advocates}{OU Advocates} (available 24/7 at 405-615-0013) or another confidential resource (see \href{https://www.ou.edu/gec/gender-based-violence/learn-more}{``Can I make an anonymous report?''}). You may also choose to report gender-based violence and discrimination through other means, including by contacting the \href{http://www.ou.edu/eoo}{Institutional Equity Office} (\href{mailto:ieo@ou.edu}{\texttt{ieo@ou.edu}}, 405-325-3546) or police (911). Because the University of Oklahoma is committed to the safety of you and other students, I, as well as other faculty, Graduate Assistants, and Teaching Assistants, are mandatory reporters. This means that we are obligated to report gender-based violence that has been disclosed to us to the Institutional Equity Office. This includes disclosures that occur in: class discussion, writing assignments, discussion boards, emails and during Student/Office Hours. For more information, please visit the \href{http://www.ou.edu/eoo}{Institutional Equity Office}.

\subsection*{Adjustments for Pregnancy/Childbirth Related Issues}

Should you need modifications or adjustments to your course requirements because of documented pregnancy-related or childbirth-related issues, please contact me or the Accessibility and Disability Resource Center at (405) 325-3852 as soon as possible. Also, see the Institutional Equity Office's \href{http://www.ou.edu/eoo/faqs/pregnancy-faqs.html}{FAQ on Pregnant and Parenting Students' Rights} for answers to commonly asked questions.

\section*{Class Schedule}
The course schedule can be found below and will be updated regularly; please check there for the most recent version.

\begin{table}[h!]
  \begin{center}
    \label{tab:table1}
    \begin{tabular}{l|c|c|c} % <-- Alignments: 1st column left, 2nd middle and 3rd right, with vertical lines in between
      \textbf{Class Session} & \textbf{Topic} & \textbf{Pre-class Reading} & \textbf{Assignment}\\
      \hline
      1 & Course Welcome &  \href{https://statsthinking21.github.io/statsthinking21-core-site/}{Preface} & NONE\\
      2 & An Intro to Github and Computing Environment &  \href{https://r4ds.had.co.nz/workflow-basics.html}{R for DS chapter 4,5,6} & NONE\\
      3 & Working with Data &  \href{https://statsthinking21.github.io/statsthinking21-core-site/}{Chapter 2} & Assignment 1\\
      4 & Summarizaing Data &  \href{https://statsthinking21.github.io/statsthinking21-core-site/}{Preface}  & NONE\\
      5 & Course Welcome &  tt & NONE\\
      6 & Course Welcome &  tt & NONE\\
      7 & Course Welcome &  tt & NONE\\
      8 & Course Welcome &  tt & NONE\\
      9 & Course Welcome &  tt & NONE\\
      10 & Course Welcome &  tt & NONE\\
      11 & Course Welcome &  tt & NONE\\
      12 & Course Welcome &  tt & NONE\\
      13 & Course Welcome &  tt & NONE\\
      14 & Course Welcome &  tt & NONE\\
      15 & Course Midterm &  Review Chapters 2,3,4,5,6,7 & NONE\\
      16 & Course Welcome &  tt & NONE\\
      17 & Course Welcome &  tt & NONE\\
      18 & Course Welcome &  tt & NONE\\
      19 & Course Welcome &  tt & NONE\\
      20 & Course Welcome &  tt & NONE\\
      21 & Course Welcome &  tt & NONE\\
      22 & Course Welcome &  tt & NONE\\
      23 & Course Welcome &  tt & NONE\\
      24 & Course Welcome &  tt & NONE\\
      25 & Course Welcome &  tt & NONE\\
      26 & Course Welcome &  tt & NONE\\
      27 & Course Welcome &  tt & NONE\\
      28 & Course Welcome &  tt & NONE\\
      29 & Course Welcome &  tt & NONE\\
      30 & Course Welcome &  tt & NONE\\

    \end{tabular}
  \end{center}
\end{table}
\end{document}
